\documentclass[11pt]{article}
\usepackage{graphicx}
\usepackage{amssymb}
\usepackage{amsmath}
\usepackage{natbib}
\usepackage{hyperref}

\title{Software \emph{Grundrisse}}
\author{Tom Augspurger}
\date{\today}
\begin{document}
\maketitle

Start with a distinction.  The two forks obvious at this time are:

\begin{itemize}
    \item[1.] Software patents.
    \item[2.] The open source software phenomenon.
\end{itemize}

A less obvious fork (2.7) is the standards v. control tradeoff. Starting place might be an overview of the intellectual property debate.  From there we can discuss the merits of each view.  Afterwards, we'll look into the specifics of each fork.  Finally, a connection to innovation and growth?

\section{Intellectual Property}
\label{sec:intellectual_property}

Mainly reading up on Boldrin and Levine, and Acemoglu \emph{et. al.}.

From there we'll delve into the niceties of the software industry.

\section{Boldrin and Levine (2008)}
\label{sec:boldrin_and_levine_}

``This is an attempt to cast doubt on the claim that monopoly is necessary for innovation, both as a matter of theory and as a matter of fact.'' p.436 (2)

They focus on innovation in a setting of \emph{perfect competition}.  First they consider some empirical examples that run contrary to the Schumpetarian idea of innovation, and then they present several models of innovations with perfect competition.

Their section on open-source software notes that monopoly rents are small, but are generated by scarcity of expertise (I think on the production side).

\section{Acemoglu and Akcigit (2012)}
\label{sec:acemoglu_and_akcigit_}

Searching for the optimal level and form of IP protection.  Full patent protection is sub-optimal.  Ought to be \ldots\ state-dependent (dependent on the ``technological gap'' between leader and laggers.  Consider the dynamic effects.  Larger gaps bad under a traditional (static) interpretation.\footnote{Notice that (at least in the intro) they say ``a large gap between the leader and the follower'', singular.  I wonder how competition among followers affects the result.}  Anyway they reach the opposite conclusion (how clever), when dynamic effects considered.  Large gaps should get more protection.  I guess as incentive to slight leader to innovate more.



\section{Software and Patents}
\label{sec:software_and_patents}

The ridiculousness of (many) software patents stands out here.  Those will be fun to investigate, but what does this state of affairs imply?  We can go into the difficulties developers face, Google v. Oracle, even go back to the founding fathers if we really want to.

A possible avenue for an empirical paper is to \emph{test} the implications of each theory in the context of software.  What assumptions does each make, and do they hold here?  What predictions does each make, and do the come to be?

Two notes of interest, both related to Microsoft. 1.) The arena of corporate IT is \emph{not} a market of perfect competition.  \textbf{The purchaser is not the user.} 2.) It would be fun to go through just how bad IE 5/6 were while Microsoft was so dominant.


\subsection{Theoretical Environment}
\label{sub:theoretical_environment}
  This subsection wants to decide which theoretical joist supports the software industry, i.e.  which best describes it.




\subsection{Core Competencies}
\label{sub:core_competencies}
  What are software's core competencies?  Ease of copy (+/- technological literacy).



\section{The Open Source Phenomenon}
\label{sec:the_open_source_phenomenon}

I think this line would engender much fun.  For now, though, I've only this nebulous interest.  Need to crystallize my thoughts before moving on.  Hence this \emph{grundrisse}.  Probably better for a second year paper.

Need to check on whether open-source is self-selecting an area to ``compete'' in.

\subsection{Han, Roberts, Slaughter, and Fielding (2004)}
\label{sub:han_roberts_slaughter_and_fielding_}

  

\section{Resding List}
\label{sec:resding_list}

\begin{itemize}
    \item Aghion, Harris, and Vickers (97): step-by-step innovation.
    \item Grossman and Helpman (91): Schumpterian.
    \item Arrow (62): Classic on static tradeoff of rents and incentives.
    \item Goleman (98)
    \item Moon and Sproull (02): Good programmers.
    \item von Krough et. al. (03): Good programmers.
\end{itemize}

\section{Potential Data Sources}
\label{sec:potential_data_sources}

\begin{itemize}
    \item NAICS-511210: R\&D, maybe at company level?
\end{itemize}

\end{document}  
