\documentclass{beamer}
\usepackage{graphicx}
\usepackage{amssymb}
\usepackage{amsmath}
\usepackage{bm}

\title{Slides for Acemoglue Akcigit (2012)}
\author{Caleb Floyd and Tom Augspurger}
\date{\today}

\begin{document}

\frame{\titlepage}

\section[Outline]{}
\frame{\tableofcontents}

% Start with general intro to paper
% Then apply to software?
% Then compare to Boldrin Levine
% Probably skip the partial equilibrium intro?

% 1. What is the question this paper answers?
% 2. What was the state of knowledge before this paper?
% 3. How does this paper answer the question differently?
% 4. What are the main findings?


\section{Introduction}
\label{sec:introduction}

\begin{frame}[t]\frametitle{Intro}
  \begin{itemize}
    \item Dynamic environment to study optimal \emph{state-dependent} Intellectual Property Right policy.
    \item IPR depends on technology gap in an industry (state-dependence).
    \item Motivates leader to continue innovating.
  \end{itemize}
\end{frame}

\section{Literature Review}
\label{sec:literature_review}
  \begin{itemize}
    \item Technology rungs (we saw in class)
    \item Look for predecessors of state-dependent IPR.  They claim to ``explicitly introduce'' it.
    \item Look for predecessor of dynamic effects, not just static R\&D incentives / monopoly rents tradeoff.
    \item Par. 3 pg. 5 notes some mechanism-design papers that might be worth skimming for previous research.
  \end{itemize}


\section{Model}
\label{sec:model}

\subsection{setup}
\label{sub:setup}

\subsection{Equilibrium}
\label{sub:equilibrium}

\subsection{Steady State}
\label{sub:steady_state}

\subsection{Characterization}
\label{sub:characterization}

\section{Optimal IPR}
\label{sec:optimal_ipr}


\section{Software}
\label{sec:software}



\end{document}
