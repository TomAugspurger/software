\documentclass[11pt]{article}
\usepackage{graphicx}
\usepackage{amssymb}
\usepackage{amsmath}

\title{Topic Proposal}
\author{Tom Augspurger}
\date{\today}
\begin{document}
\maketitle

We will consider several theoretical frameworks for thinking about software patents in particular.  What are the features of an ``optimal software patent system'', if such a thing exists? The two competing frameworks are Acemoglu et. al (2012) and Boldrin and Levine (2008).  Obviously, these models aren't directly comparable. It will be necessary to translate the assumptions and results of each into a common language.  

Since we are applying these models to the software industry, modifications might need to be made, additional resources may need to be culled, etc.  Likely, a chunk of time will be need to motivate the question and give some background.  Patents on software are becoming more pervasive and many stories have been written about how ridiculous some patents are.  We also have \emph{Google v. Oracle} with the Java API for Android and patent trolls --- firms who's sole method of revenue suing companies for infringing on its patents.

\end{document}
