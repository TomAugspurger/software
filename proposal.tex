\documentclass[11pt]{article}
\usepackage{graphicx}
\usepackage{amssymb}
\usepackage{amsmath}

\title{Topic Proposal}
\author{Tom Augspurger and Caleb Floyd}
\date{\today}
\begin{document}
\maketitle

We will consider several theoretical frameworks for thinking about software patents in particular.  What are the features of an ``optimal software patent system'', if such a thing exists? The two competing frameworks are Acemoglu et. al (2012) and Boldrin and Levine (2008).  Obviously, these models aren't directly comparable. It will be necessary to translate the assumptions and results of each into a common language. Other frameworks will be considered as necessary.

We'll likely need to devote considerable time describing the landscape of intellectual property rights in the software industry, converting legalese into workable language, and motivating the conversation. To this extent case studies should provide ample ammunition. Some possible examples are \emph{Google v. Oracle} with the Java API for Android lawsuit and patent trolls --- firms who's sole method of revenue suing companies for infringing on its patents.

If time permits, empirical papers will also be studied for relevance: Wu et al. $(2007)$ and Hall \& MacGarive $(2010)$ are two such examples that offer empirical analysis of patents in the software industry.  These will indicate if either framework is able to describe the software industry.  Which assumptions hold? Given those assumptions, which predictions are likely to be true?
\end{document}
