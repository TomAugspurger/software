\documentclass[11pt]{article}
\usepackage{graphicx}
\usepackage{amssymb}
\usepackage{amsmath}

\title{Topic Proposal}
\author{Tom Augspurger}
\date{\today}
\begin{document}
\maketitle

We will consider several theoretical frameworks for thinking about software patents in particular.  What are the features of an ``optimal software patent system'', if such a thing exists? The two main contenders are a Acemoglu et. al (2012) and Boldrin and Levine (2008).  Obviously, these models aren't directly comparable.  One assumes perfect competition while the other allows for (and encourages to an extent) monopoly. It will be necessary to translate the assumptions and results of each into a common language.  

Since we are applying these models to the software industry, modifications might need to be made, additional resources may need to be culled, etc.  Likely, a chunk of time will be need to motivate the question and give some background.  One possible extension (or inclusion if the scope is manageable) is to examine open source software and its forfeit of any kind of intellectual monopoly.
\end{document}
