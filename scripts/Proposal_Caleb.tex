\documentclass[11pt]{article}
\usepackage{graphicx}
\usepackage{amssymb}
\usepackage{amsmath}

%%%%%%%%%%%%%%%%%%%%%%%%%%%%%%%%%%%%%%%%%%%%%%%%%%%%%%%%%%

\title{Proposal}
\author{Caleb Floyd}
\date{\today}

\begin{document}
\maketitle

The goal of this survey is to analyze frameworks of intellectual property rights as they pertain to growth and the software industry. As of now, the obvious frameworks are Boldrin \& Levine $(2008)$ and Acemoglu \& Akcigit $(2012)$. To the extent that these frameworks live in different market structures, we'll need to undertake careful analysis of the software industry to see when and where each framework applies, if either applies. Other frameworks may also need to be studied in some cases. 

We'll likely need to devote considerable time describing the landscape of intellectual property rights in the software industry, converting legalese into workable language, and motivating the conversation. To this extent case studies should provide ample ammunition. 

Empirical papers will also be studied for relevance: Wu et al. $(2007)$ and Hall \& MacGarive $(2010)$ are two such examples that offer empirical analysis of patents in the software industry. Clearly, other work may need to be studied as well. It is our conceit that debate on proprietary vs. open source software will provide a rich environment in which to analyze the impact of patent protection on the industry, and hopefully engender pertinent questions for future work. 

\end{document}