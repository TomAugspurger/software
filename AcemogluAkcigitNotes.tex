\documentclass[11pt]{article}
\usepackage{graphicx}
\usepackage{amssymb}
\usepackage{amsmath}

\title{Notes on Acemoglu and Akcigit}
\author{Caleb Floyd and Tom Augspurger}
\date{\today}
\begin{document}
\maketitle

\section{The Model}
\label{sec:the_model}

\subsection{Assumptions}
\label{sub:assumptions}

  \begin{itemize}
    \item \textbf{State dependent patent policy:} $\mathbf{\eta} : \mathbf{N}_I \rightarrow \mathbb{R_+}$
    \item (Temporary) Leader is a max of two ``steps'' ahead.
    \item (Temporary) \textbf{Instant catch-up once patent expires.}
    \item Patents expire with a Poisson rate $\eta$ (otherwise time since innovation becomes a state variable in the value functions.)
    \item (Temporary) Cost of R\&D is linear in innovation, $\Phi(x_i) = \phi x_i$.
    \item (Temporary?) Interest rate $r$ is exogenous with some parameter restion to ensure positive R\&D by each firm when they are tied.
    \item logarithmic utility
    \item Cobb-Douglas production
    \item \textbf{free-entry into \emph{final} goods market.}
    \item \textbf{Two varieties of intermediates that are perfect substitutes and compete in Bertrand style.}
    \item Consumers own balanced portfolio of intermediate producers.
    \item (Temporary?) Catch-up and leader R\&D have the same costs and success probability.
    \item Innovation technology $F \in C^2$, $F'(\,\cdot\,) > 0, F''(\,\cdot\,) < 0, F'(0) < \infty$, and $\ \exists\ \bar{h} \in (0, \infty) : F'(h) = 0 \forall h \geq \bar{h}$.
  \end{itemize}

\subsection{Preferences}
\label{sub:preferences}

  Continuum of individuals (mass 1?) with 1 unit of labor supplied inelastically.

  \begin{equation} \label{eq:pref}
    \mathbb{E}_t \int_t^\infty exp(-\rho(s - t))\mathrm{ln}\,   C(s)ds
  \end{equation}

  where $\rho$ is a discount rate and $C(t)$ is consumption at $t$.

\subsection{Production}
\label{sub:production}

  In the partial equilibrium setup, firms maximize

  \begin{equation}
    E_t \int_t^\infty exp(-r(s - t))[\pi_i(s) - \Phi_i(s)]ds
  \end{equation}

  where $r > 0$ is the interest rate, $\pi_i(t)$ is the instantaneous profit, $\Phi_i(t)$ is the R\&D cost.

  Euler equation:

  \begin{equation}
    g(t) \equiv \frac{\dot{C}(t)}{C(t)} = \frac{\dot{Y}(t)}{Y(t)} = r(t) - \rho
  \end{equation}

  where $g(t)$ is growth rate of consumption / output, $r(t)$ is the interest rate.
  Innovation follows a Poisson Process with rate of arrival $x_i$.

  In the General Equilibrium setup, there's a continuum of intermediate goods with a CD production, differentiated into two varieties, each produced by a single firm.

  \begin{equation} \label{eq:tech_output}
    \ln Y(t) = \int_{0}^{1} \ln y(j, t) d\,j 
  \end{equation}

  with individual production of good $j$ by firm $i$ at time $t$ as:

  \begin{equation}
    y(j, t) = q_i(j, t)l_i(j, t)
  \end{equation}

  where $q_i$ is a technology level and $l_i$ is labor used.

  Speaking of technology, every firm has the R\&D technology:

  \begin{equation} \label{eq:tech_rd_technology}
    x_i(j, t) = F(h_i(j, t))
  \end{equation}

  where $x_i$ is the flow rate of innovation and $h_i$ is the number of workers.  This means the cost of R\&D is $w(t)G(x_i(j,t)$ where $G(x_i(j,t)) \equiv F^{-1}(x_i(j,t))$.

  Law of motion for technology gap in industry $j$:

  \begin{equation} \label{eq:tech_law_of_motion}
    \eta_j(t + \Delta t) =
    \begin{cases}
      \eta_j(t) + 1 & \textrm{prob } x_i(j,t)\Delta t + o(\Delta t)\\
      0 & \textrm{prob } x_{-i}(j,t)\Delta t + \eta_{n_{j(t)}}\Delta t + o(\Delta t) \\
      \eta_j(t) & \textrm{prob } 1 - x_i(j,t)\Delta t + x_{-i}(j,t)\Delta t + \eta_{n_{j(t)}}\Delta t - o(\Delta t)
    \end{cases}
  \end{equation}

\end{document}  
