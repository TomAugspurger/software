\documentclass[11pt]{article}
\usepackage{graphicx}
\usepackage{amssymb}
\usepackage{amsmath}
\usepackage{bm}

\title{Notes on Acemoglu and Akcigit}
\author{Caleb Floyd and Tom Augspurger}
\date{\today}
\begin{document}
\maketitle

\section{The Model}
\label{sec:the_model}

\subsection{Assumptions}
\label{sub:assumptions}

  \begin{itemize}
    \item IPR Policy
    \begin{itemize}
      \item \textbf{State dependent patent policy:} $\bm{\eta} : \bm{N}_I \rightarrow \mathbb{R_+}$      
    \end{itemize}

    \item Technology
    \begin{itemize}
      \item (Temporary) Leader is a max of two ``steps'' ahead.
      \item (Temporary) \textbf{Instant catch-up once patent expires.}
      \item Patents expire with a Poisson rate $\eta$ (otherwise time since innovation becomes a state variable in the value functions.)
      \item \textbf{Linear Costs important for Proposition 1}.
      \item \textbf{Cobb-Douglas production} (``implies profits depend only on the technology gap of the industry and aggregate output. (page 14)'')
      \item \textbf{free-entry into \emph{final} goods market.}  
      \item \textbf{Two varieties of intermediates that are perfect substitutes and compete in Bertrand style.}
      \item (Temporary?) Catch-up and leader R\&D have the same costs and success probability.
      \item Innovation technology $F \in C^2$, $F'(\,\cdot\,) > 0, F''(\,\cdot\,) < 0, F'(0) < \infty$, and $\ \exists\ \bar{h} \in (0, \infty) : F'(h) = 0 \forall h \geq \bar{h}$.
    \end{itemize}
    
    \item Preferences
    \begin{itemize}
      \item (Temporary?) Interest rate $r$ is exogenous with some parameter restion to ensure positive R\&D by each firm when they are tied.
      \item logarithmic utility
      \item Consumers own balanced portfolio of intermediate producers. 
    \end{itemize}
    
    \item Equilibrium
    \begin{itemize}
      \item Only focus on Markov perfect Equilibria (strategies are functions only of payoff-relevant state variables.)
      \item $G'^{-1}((1 0 \lambda^{-1}) / (\rho + \eta)) > 0$ Implies positive RND in equilibrium.
    \end{itemize}
    
  \end{itemize}

\subsection{Preferences}
\label{sub:preferences}

  Continuum of individuals with 1 unit of labor supplied inelastically.

  \begin{equation} \label{eq:pref}
    \mathbb{E}_t \int_t^\infty exp(-\rho(s - t))\mathrm{ln}\,   C(s)ds
  \end{equation}

  where $\rho$ is a discount rate and $C(t)$ is consumption at $t$.

\subsection{Production}
\label{sub:production}

  In the partial equilibrium setup, firms maximize

  \begin{equation}
    E_t \int_t^\infty exp(-r(s - t))[\pi_i(s) - \Phi_i(s)]ds
  \end{equation}

  where $r > 0$ is the interest rate, $\pi_i(t)$ is the instantaneous profit, $\Phi_i(t)$ is the R\&D cost.

  Euler equation:

  \begin{equation}
    g(t) \equiv \frac{\dot{C}(t)}{C(t)} = \frac{\dot{Y}(t)}{Y(t)} = r(t) - \rho
  \end{equation}

  where $g(t)$ is growth rate of consumption / output, $r(t)$ is the interest rate.
  Innovation follows a Poisson Process with rate of arrival $x_i$.

  In the General Equilibrium setup, there's a continuum of intermediate goods with a CD production, differentiated into two varieties, each produced by a single firm.

  \begin{equation} \label{eq:tech_output}
    \ln Y(t) = \int_{0}^{1} \ln y(j, t) d\,j 
  \end{equation}

  with individual production of good $j$ by firm $i$ at time $t$ as:

  \begin{equation}
    y(j, t) = q_i(j, t)l_i(j, t)
  \end{equation}

  where $q_i$ is a technology level and $l_i$ is labor used.

  Speaking of technology, every firm has the R\&D technology:

  \begin{equation} \label{eq:tech_rd_technology}
    x_i(j, t) = F(h_i(j, t))
  \end{equation}

  where $x_i$ is the flow rate of innovation and $h_i$ is the number of workers.  This means the cost of R\&D is $w(t)G(x_i(j,t)$ where $G(x_i(j,t)) \equiv F^{-1}(x_i(j,t))$.

  Law of motion for technology gap in industry $j$:

  \begin{equation} \label{eq:tech_law_of_motion}
    \eta_j(t + \Delta t) =
    \begin{cases}
      \eta_j(t) + 1 & \textrm{prob } x_i(j,t)\Delta t + o(\Delta t)\\
      0 & \textrm{prob } x_{-i}(j,t)\Delta t + \eta_{n_{j(t)}}\Delta t + o(\Delta t) \\
      \eta_j(t) & \textrm{prob } 1 - x_i(j,t)\Delta t + x_{-i}(j,t)\Delta t + \eta_{n_{j(t)}}\Delta t - o(\Delta t)
    \end{cases}
  \end{equation}

  We also have the operating profits (i.e. before R\&D) for the firm:
  
  \begin{align*} \label{eq:profits}
    \Pi_i(j, t) &= [p_i(j, t) - MC_i(j, t)]y_i(j, t)\\
                &= (1 - \lambda^{\eta_j(t)})Y(t)
  \end{align*}

  first line is price - marginal cost times quantity.  And the second line relates profites to the technology gap and aggregate output.  This should just be the leader who is selling in the industry, but I need to double check that.

\subsection{Equilibrium}
\label{sub:equilibrium}
  $\bm{\mu}(t) \equiv {\mu_n(t)}_n^\infty$ is a distribution of \emph{industries} over \emph{technology gaps} (and not the other way around).  This means that (I think), $\bm{\mu}(t)$ is a valid pdf.

  $\xi_n(t)$ is a list of decisions for the leader with a technology gap $n$.  Includes R\&D, price, and output for the leader and just R\&D for the follower.  Footnote 15 notes that the choices for price output depend on the actual level of the technology $x_{ij}(t)$, not just the gap.  But profits (p * y) are independent of the level, and just depend on the gap.

  For an optimal policy $\hat{x}$ the leader in some industries with a lead of $n$ steps has a present discounted value of 

  \begin{equation}
    V_n(t) = \mathbb{E}_t \int_{t}^{\infty} exp(-r(s - t))[\Pi(s) - w(s)G(\hat{x}(s))]ds
  \end{equation}
  with $\Pi(s)$ operating profits and the last term is the wage bill on R\&D.  I think this is just a single industry so there will be another integral over all industries (whose distribution is $\bm{\mu}$).

  Here we have the central optimization problems.

  Leaders:

  \begin{equation} \label{eq:rvf_leader}  % rvf: recursive value function
    pv_n = \max_{x_n \geq 0} (1 - \lambda^{-n}) - \omega^*G(x_n) + x_n[v_{n+1} - v_n] + [x_{-n}^* + \eta_n][v_0 - v_n]
  \end{equation}

  Tied:

  \begin{equation} \label{eq:rvf_tied}
    \rho v_0 = \max_{x_0 \geq 0} -\omega^*G(x_{0} + x_{0}[v_1 - v_0] + x_0^*[v_{-1} - v_0]
  \end{equation}

  and followers:

  \begin{equation} \label{eq:rvf_follower}
    \rho v_{n-1} \max_{x_n \geq 0} -\omega^*G(x_{-n} + [x_{-n} + \eta_n][v_0 - v_{-n}] + x_n^*[v_{-n-1} - v_{-n}])
    \end{equation}

  Whose solutions satisfy:

  \begin{align} \label{eq:ss_rd_policies}
    x_n^*    &= max \big\{G'^{-1}\frac{[v_{n+1} - v_n]}{\omega^*}   ,0\}  
    x_{-n}^* &= max \big\{G'^{-1}\frac{[v_0  - v_{-n}]}{\omega^*}   ,0\}
    x_0^*    &= max \big\{G'^{-1}\frac{[v_1     - v_0]}{\omega^*}   ,0\}
  \end{align}
  
  which says that innovation rates increase when the value of moving ahead another step is larger or the cost of R\&D is ($\omega*$) is lower.  Go on to discuss some \emph{positive} incentive effects from \emph{weakening} patent protection (p18).

  From here they use the steady state distribution of leads over industries to equal the flows into a given lead (leader innovating) to the flows out (catch up or patent expiry).


\subsection{Steady-State Equilibria}
\label{sub:steady_state_equilibria}
  
  \subsubsection{Uniform IPR}
  \label{subsub:uniform_ipr}
    Get some results on existence with positive RND.  Find $\mathbb{x}^* = \{x_n^*\}_n=1^{\infty}$ is \emph{decreasing} so R\&D is decreasing in the technology gap via smaller increase in the markup gap for each lead increase.  Also get highest R\&D when firms are tied.

  \subsubsection{State-Dependent IPR}
  \label{state_dependent_ipr}
    Get existence. No analytical results.

    Define welfare at date zero.  Function of date zero output, $\rho$ (discount factor), and $g^*$ (growth rate of wages I think? Or output or the quality index?). Take two identical date-0 economies and set them off to steady state under different IPR policies.  Key parameter here is $Q(t)$ the quality index.

    Two types of effects:
    \begin{itemize}
      \item Growth Effect: via growth rate.  Is $g^*/\rho^2$.
      \item Distribution Effect: $\ln(\sum_{n=0}^{\infty} n\mu_n)/\rho - \ln\omega(0)$.
    \end{itemize}

    Growth effects will dominate (as they tend to do).  For a variety of reasons this equilibrium is inefficient.

\subsection{Optimal IPR Policy, Quantitative Results}
\label{sub:optimal_ipr_policy_quantitative_results}

  One possible interpretation of state-contingent IPR policy is if you get too far ahead the government comes in a ends your monopoly.

  Under uniform IPR, they find full protection ($\eta = 0$) is best.  Driven by the quick catch-up assumption: firms spend a lot of time tied, where innovation is largest.

  Under State-Dependent IPR, patent length should be increasing in technology gap.  Composition effect of most innovation when firms are tied is swamped by a trickle-down effect (small leaders have incentive to innovate to become larger leaders).

\subsection{Slow Catch Up}
\label{sub:slow_catch_up}

  Did full protection and didn't like the results(?) so they change $\eta$ to 0.02. Only 8\% of firms within 2 steps of each other (rampant monopoly).

  Under uniform IPR, full-protection is no longer optimal.  I wish they had done no policy protection. 

  Similar findings for State-Dependent IPR as above.  $\eta$ decreasing in gap.

  \subsubsection{Compulsory Licensing}
  \label{subsub:compulsory_licensing}

    Results are mostly unchanged.  Optimal uniform IPR is better than full IPR, but state-dependent is better for similar reasons.

  \subsubsection{Leapfrogging}
  \label{subsub:leapfrogging}

    More sensibly, follower can advance growth of economy via research.  For some reason they introduce a probability of infringement if they \emph{pass} the other firm, but not if the \emph{equal} the firm.  This probability is also state-contingent.  Same results.

\section{Thoughts}
\label{sec:thought}

  Is treating innovation processes as independent a good idea?  You could tell a story where increased competition drives innovation rates for everyone (technology spillovers, agglomeration), while a large gap decrease innovation for laggards (financing, prestige) and for leaders (stagnation).

  Should have investigated a probability of infringement by firms who catch up.  Reduces incentive to get to catch-up point, so lower growth presumably.
\end{document}  
